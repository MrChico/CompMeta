\documentclass{article}
\usepackage{amsmath}
\usepackage{proof}

\begin{document}
\title{Computational Metaphysics 1}
\author{Denise Erfurt, Martin Lundfall}
\maketitle
\section*{Exercise 4}
\subsection*{a)}
\begin{equation}
  \infer[\mathrm{impI}]
        {A \rightarrow A}
        {\infer[\mathrm{id}]{A}{[A]}}        
\end{equation}
\subsection*{b)}
\begin{equation}
  \infer[\mathrm{impI}]
        {A \rightarrow (B \rightarrow A)}
        {\infer[\mathrm{impI}]
          {B \rightarrow A}
          {\infer[\mathrm{id}]
            {A}
            {[A]}
        }
   }
\end{equation}
Note that A follows independently of B, so in particular, it follows from B. We can always add arbitrary assumptions, even if our conclusions do not need them.
\subsection*{c)}
\begin{equation}
  \infer[\mathrm{impI_2}]
        {(A \rightarrow (B \rightarrow C)) \rightarrow ((A \rightarrow B) \rightarrow (A \rightarrow C))}
        {\infer[\mathrm{impI_3}]
          {(A \rightarrow B) \rightarrow (A \rightarrow C)}
          {\infer[\mathrm{impI_1}]
            {A \rightarrow C}
            {\infer[\mathrm{mp}]
              {C}
              {
                \infer[\mathrm{mp}]{B}{\infer[\mathrm{id}]{A}{[A]^1} & [A \rightarrow B]^3} &
                \infer[\mathrm{mp}]{B \rightarrow C}
                {\infer[\mathrm{id}]{A}{[A]^1} & [A \rightarrow (B \rightarrow C)]^2}
              }
            }
          }
        }
\end{equation}
\subsection*{d)}
\begin{equation}
  \infer[\mathrm{impI_3}]
        {( \neg A \rightarrow \neg B) \rightarrow (B \rightarrow A)}
        {\infer[\mathrm{impI_1}]
          {B \rightarrow A}
          {\infer[\mathrm{ccontr_2}]
            {A}
            {\infer[\mathrm{notE}]
              {\bot}
              {[B]^1 &
                \infer[\mathrm{mp}]
                      {\neg B}
                      {[\neg A]^2 & [\neg A \rightarrow \neg B]^3}
              }
            }
          }
        }
\end{equation}
\end{document}
