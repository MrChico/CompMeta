



\documentclass[11pt,a4paper]{article}
\usepackage{isabelle,isabellesym}

% further packages required for unusual symbols (see also
% isabellesym.sty), use only when needed

%\usepackage{amssymb}
  %for \<leadsto>, \<box>, \<diamond>, \<sqsupset>, \<mho>, \<Join>,
  %\<lhd>, \<lesssim>, \<greatersim>, \<lessapprox>, \<greaterapprox>,
  %\<triangleq>, \<yen>, \<lozenge>

%\usepackage{eurosym}
  %for \<euro>

%\usepackage[only,bigsqcap]{stmaryrd}
  %for \<Sqinter>

%\usepackage{eufrak}
  %for \<AA> ... \<ZZ>, \<aa> ... \<zz> (also included in amssymb)

%\usepackage{textcomp}
  %for \<onequarter>, \<onehalf>, \<threequarters>, \<degree>, \<cent>,
  %\<currency>

% this should be the last package used
\usepackage{pdfsetup}
\usepackage{bookmark}
\usepackage{proof}
\usepackage{MnSymbol}
\usepackage{hyperref}

% urls in roman style, theory text in math-similar italics
\urlstyle{rm}
\isabellestyle{it}

% for uniform font size
%\renewcommand{\isastyle}{\isastyleminor}


\begin{document}

\title{Investigations of Spatial Logic}
\author{By Martin Lundfall, Denis Erfurt}
\maketitle

\begin{abstract}
  This is a formalization of a reflective, higher order process calculus known as the
  rho-calculus and its associated logic, namespace logic.
  The calculus is described by Greg Meredith in detail in the following resources:\\
  \begin{itemize}
  \item \href{http://www.lshift.net/downloads/ex_nihilo_logic.pdf}{Namespace Logic}
  \item \href{https://arxiv.org/pdf/1307.7766v2.pdf}{Policy as Types}
  \end{itemize}
\end{abstract}
\newpage
\tableofcontents

% sane default for proof documents
\parindent 0pt\parskip 0.5ex

% generated text of all theories
\input{session}

% optional bibliography
%\bibliographystyle{abbrv}
%\bibliography{root}

\end{document}

%%% Local Variables:
%%% mode: latex
%%% TeX-master: t
%%% End:
