% Article template for Mathematics Magazine
% Revised 7/2002  Thanks for Greg St. George
\documentclass[12pt]{article}
\usepackage{amssymb}
\usepackage[ngerman]{babel}
\usepackage[utf8]{inputenc}
\usepackage{amsmath}
\usepackage{pf2}
\usepackage{graphicx}
\renewcommand{\baselinestretch}{1.2}
%This is the command that spaces the manuscript for easy reading
\newtheorem{zeige}{Zeige}


%todo 
\usepackage[colorinlistoftodos,prependcaption,textsize=tiny]{todonotes}
\usepackage{xargs}                      % Use more than one optional parameter in a new commands
\newcommandx{\QUESTION}[2][1=]{\todo[linecolor=none,backgroundcolor=blue!15,bordercolor=none,#1]{\textbf{QUESTION: }#2}}

\pflongnumbers
\def\assumePfkwd{\textsc{Assume}:}%
\def\provePfkwd{\textsc{Zeige}:}%
\def\pickPfkwd{\textsc{Pick}}%
\def\pfnewPfkwd{\textsc{New}}%
\def\casePfkwd{\textsc{Fall}:}%
\def\letPfkwd{\textsc{Sei}:}%
\def\sufficesPfkwd{\textsc{Suffices}:}%
\def\asufficesPfkwd{\textsc{Suffices}}%
\def\definePfkwd{\textsc{Def.}:}%
\def\proofPfkwd{\textsc{Beweis}:}%
\def\proofsketchPfkwd{\textsc{Proof sketch}:}%
\def\qedPfkwd{{\fboxsep=\z@\fbox{\rule{.5em}{0pt}\rule{0pt}{2ex}}}}
\def\qedstepPfkwd{Q.E.D.}



\begin{document}
%\thispagestyle{empty}
\begin{center}
\Large
% TITLE GOES HERE
Logik und Komplexität  \textsc{ Übung 7 }
\end{center}

\begin{flushright}
Denis Erfurt, 532437\\
HU Berlin \\

\vspace{2 mm}

\end{flushright}





\subsection*{Excercise 1)}
\subsubsection*{a)}
”The ship is huge and it is blue.”
\begin{equation}
  Huge(the\_ship) \land Blue(the\_ship)
\end{equation}
\subsubsection*{b)}
”I’m sad if the sun does not shine.”
\begin{equation}
  \neg Sun\_is_\_shining \rightarrow Sad(I)
\end{equation}
\subsubsection*{c)}
”Either it’s raining or it is not.”
\begin{equation}
  Is\_raining \lor Is\_not\_raining
\end{equation}
\subsubsection*{d)}
”I’m only going if she is going!”
\begin{equation}
  I\_am\_going \leftrightarrow She\_is\_going
\end{equation}
\subsubsection*{e)}
”Everyone loves chocolate or ice cream.”
\begin{equation}
  \forall x Is\_someone(x) \rightarrow  Loves\_ice\_cream(x) \lor Loves\_chocolate(x)
\end{equation}
\subsubsection*{f)}
”There is somebody who loves ice cream and loves chocolate as well.”
\begin{equation}
  \exists x Is\_someone(x) \land Loves\_ice\_cream(x) \land Loves\_chocolate(x)
\end{equation}
\subsubsection*{g)}
”Everyone has got someone to play with.”
\begin{equation}
  \forall x \exists y Is\_someone(x) \land Is\_someone(y) \land Can\_play\_with(x, y)
\end{equation}
\subsubsection*{h)}
”Nobody has somebody to play with if they are all mean.”
\begin{equation}
  \neg \exists x \exists y Is\_someone(x) \land Is\_somenone(y) \land Is\_mean(x) \land Can\_play\_with(x, y)
\end{equation}
\subsubsection*{i)}
”Cats have the same annoying properties as dogs.”
\begin{equation}
  \forall P \forall cat \forall dog (Is\_annoying(P) \land Is\_cat(cat) \land Is\_dog(dog)) \rightarrow  (P(cat) \leftrightarrow P(dog)))
\end{equation}
\subsection*{Excercise 2)}
\subsubsection*{a)}
\subsubsection*{b)}
\subsection*{Excercise 3)}
\subsection*{Excercise 4)}





\end{document}
