% Article template for Mathematics Magazine
% Revised 7/2002  Thanks for Greg St. George
\documentclass[12pt]{article}
\usepackage{amssymb}
\usepackage[ngerman]{babel}
\usepackage[utf8]{inputenc}
\usepackage{amsmath}
\usepackage{pproof}
\usepackage{fitch}
\usepackage{pf2}
\usepackage{graphicx}
\renewcommand{\baselinestretch}{1.2}
%This is the command that spaces the manuscript for easy reading
\newtheorem{zeige}{Zeige}


%todo 
\usepackage[colorinlistoftodos,prependcaption,textsize=tiny]{todonotes}
\usepackage{xargs}                      % Use more than one optional parameter in a new commands
\newcommandx{\QUESTION}[2][1=]{\todo[linecolor=none,backgroundcolor=blue!15,bordercolor=none,#1]{\textbf{QUESTION: }#2}}

\pflongnumbers
\def\assumePfkwd{\textsc{Assume}:}%
\def\provePfkwd{\textsc{Zeige}:}%
\def\pickPfkwd{\textsc{Pick}}%
\def\pfnewPfkwd{\textsc{New}}%
\def\casePfkwd{\textsc{Fall}:}%
\def\letPfkwd{\textsc{Sei}:}%
\def\sufficesPfkwd{\textsc{Suffices}:}%
\def\asufficesPfkwd{\textsc{Suffices}}%
\def\definePfkwd{\textsc{Def.}:}%
\def\proofPfkwd{\textsc{Beweis}:}%
\def\proofsketchPfkwd{\textsc{Proof sketch}:}%
\def\qedPfkwd{{\fboxsep=\z@\fbox{\rule{.5em}{0pt}\rule{0pt}{2ex}}}}
\def\qedstepPfkwd{Q.E.D.}



\begin{document}
%\thispagestyle{empty}
\title{Computational Metaphysics 1}
\author{Denise Erfurt, Martin Lundfall}
\maketitle





\subsection*{Excercise 1)}
\subsubsection*{a)}
”The ship is huge and it is blue.”
\begin{equation}
  Huge(the\_ship) \land Blue(the\_ship)
\end{equation}
\subsubsection*{b)}
”I’m sad if the sun does not shine.”
\begin{equation}
  \neg Sun\_is_\_shining \rightarrow Sad(I)
\end{equation}
\subsubsection*{c)}
”Either it’s raining or it is not.”
\begin{equation}
  Is\_raining \lor Is\_not\_raining
\end{equation}
\subsubsection*{d)}
”I’m only going if she is going!”
\begin{equation}
  I\_am\_going \leftrightarrow She\_is\_going
\end{equation}
\subsubsection*{e)}
”Everyone loves chocolate or ice cream.”
\begin{equation}
  \forall x Is\_someone(x) \rightarrow  Loves\_ice\_cream(x) \lor Loves\_chocolate(x)
\end{equation}
\subsubsection*{f)}
”There is somebody who loves ice cream and loves chocolate as well.”
\begin{equation}
  \exists x Is\_someone(x) \land Loves\_ice\_cream(x) \land Loves\_chocolate(x)
\end{equation}
\subsubsection*{g)}
”Everyone has got someone to play with.”
\begin{equation}
  \forall x \exists y Is\_someone(x) \land Is\_someone(y) \land Can\_play\_with(x, y)
\end{equation}
\subsubsection*{h)}
”Nobody has somebody to play with if they are all mean.”
\begin{equation}
  \forall x Is\_someone(x) \land Is\_mean(x) \rightarrow  \neg \exists y \land Is\_somenone(y)\land Can\_play\_with(x, y)
\end{equation}
\subsubsection*{i)}
”Cats have the same annoying properties as dogs.”
\begin{equation}
  \forall P \forall cat \forall dog (Is\_annoying(P) \land Is\_cat(cat) \land Is\_dog(dog)) \rightarrow  (P(cat) \leftrightarrow P(dog)))
\end{equation}
\subsection*{Excercise 2)}
\subsubsection*{a)}
propositional
\subsubsection*{b)}
higher-order
\subsubsection*{c)}
first-order
\subsubsection*{d)}
higher-order
\subsection*{Excercise 3)}
\subsection*{a)}
\begin{equation}
  A \land B \rightarrow C, B \rightarrow A, B \vdash C
\end{equation}

\begin{fitch}
B & $ass.$ \\
B \rightarrow A & $ass.$ \\
A & $mp(1, 2)$ \\
A \land B & $conjI(1,3)$ \\
A \land B \rightarrow C & $ass.$ \\
C & $mp(4,5)$
\end{fitch}

\subsection*{b)}
\begin{equation}
  A \vdash B \rightarrow A
\end{equation}
\begin{fitch}
  A & $ass.$ \\
  \fh B & $hyp.$ \\
  \fa B \rightarrow A & $impI(1,2)$
\end{fitch}
\subsection*{c)}
\begin{equation}
  A \rightarrow (B \rightarrow C) \vdash B \rightarrow (A \rightarrow C)
\end{equation}
\begin{equation}
\begin{fitch}
  \fh B & $hyp.$ \\
  \fa \fh A & $hyp.$ \\
  \fa \fa A \rightarrow (B \rightarrow C) & $ass.$ \\
  \fa \fa B \rightarrow C & $mp(2,3)$ \\
  \fa C & $mp(1,4)$ \\
  \fa A \rightarrow C & $impI(2,5)$ \\
  \fa B \rightarrow (A \rightarrow C) & $impI(1,6)$
\end{fitch}
\end{equation}
\subsection*{d)}
\begin{equation}
  \neg A \vdash A \rightarrow B
\end{equation}
\begin{equation}
\begin{fitch}
  \neg A & $ass.$ \\
  \fh A & $hyp.$ \\
  \fa \fh \neg B & $hyp$ \\
  \fa \fa \bot & $notE(1,2)$ \\
  \fa B & $ccontr(2,4)$\\
  \fa A \rightarrow B & $imp(2,5)$
\end{fitch}
\end{equation}
\subsection*{e)}
\begin{equation}
  \vdash A \lor \neg A
\end{equation}
\begin{equation}
\begin{fitch}
  \fh \neg (A\lor \neg A) & $hyp.$\\
  \fa \fh A & $hyp.$\\
  \fa \fa A \lor \neg A & $disjI(2)$ \\
  \fa \bot & $notE(1,3)$ \\
  A \lor \neg A & $ccontr(1,4)$
\end{fitch}
\end{equation}
% \subsection*{f)}
% \begin{equation}
%   A \lor B, \neg A \vdash B
% \end{equation}
% \subsection*{g)}
% \begin{equation}
%   \neg A \lor B \vdash A \rightarrow B
% \end{equation}


\subsection*{Excercise 4)}
\subsection*{a)}
\begin{equation}
  \infer[\mathrm{impI}]
        {A \rightarrow A}
        {\infer[\mathrm{id}]{A}{[A]}}
\end{equation}
\subsection*{b)}
\begin{equation}
  \infer[\mathrm{impI}]
        {A \rightarrow (B \rightarrow A)}
        {\infer[\mathrm{impI}]
          {B \rightarrow A}
          {\infer[\mathrm{id}]
            {A}
            {[A]}
        }
   }
\end{equation}
Note that A follows independently of B, so in particular, it follows from B. We can always add arbitrary assumptions, even if our conclusions do not need them.
\subsection*{c)}
\begin{equation}
  \infer[\mathrm{impI_2}]
        {(A \rightarrow (B \rightarrow C)) \rightarrow ((A \rightarrow B) \rightarrow (A \rightarrow C))}
        {\infer[\mathrm{impI_3}]
          {(A \rightarrow B) \rightarrow (A \rightarrow C)}
          {\infer[\mathrm{impI_1}]
            {A \rightarrow C}
            {\infer[\mathrm{mp}]
              {C}
              {
                \infer[\mathrm{mp}]{B}{\infer[\mathrm{id}]{A}{[A]^1} & [A \rightarrow B]^3} &
                \infer[\mathrm{mp}]{B \rightarrow C}
                {\infer[\mathrm{id}]{A}{[A]^1} & [A \rightarrow (B \rightarrow C)]^2}
              }
            }
          }
        }
\end{equation}
\subsection*{d)}
\begin{equation}
  \infer[\mathrm{impI_3}]
        {( \neg A \rightarrow \neg B) \rightarrow (B \rightarrow A)}
        {\infer[\mathrm{impI_1}]
          {B \rightarrow A}
          {\infer[\mathrm{ccontr_2}]
            {A}
            {\infer[\mathrm{notE}]
              {\bot}
              {[B]^1 &
                \infer[\mathrm{mp}]
                      {\neg B}
                      {[\neg A]^2 & [\neg A \rightarrow \neg B]^3}
              }
            }
          }
        }
\end{equation}



\end{document}
